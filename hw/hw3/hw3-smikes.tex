\documentclass[12pt,letterpaper,fleqn]{hmcpset}
\usepackage[margin=1in]{geometry}
\usepackage{graphicx}
\usepackage{amsmath,amssymb}
\usepackage{enumerate}
\usepackage{hyperref}
\usepackage{parskip}

\input{macros.tex}

% info for header block in upper right hand corner
\name{Sam Mikes}
\class{Math189R SU20.1}
\assignment{Homework 3}
\duedate{Thursday, June 18, 2019}

\begin{document}

\begin{problem}[1]
(\textbf{Murphy 2.16}) Suppose $\theta \sim \text{Beta}(a,b)$ such
        that
        \[
            \PP(\theta; a,b) = \frac{1}{B(a,b)} \theta^{a-1}(1-\theta)^{b-1} = \frac{\Gamma(a+b)}{\Gamma(a)\Gamma(b)} \theta^{a-1}(1-\theta)^{b-1}
        \]
        where $B(a,b) = \Gamma(a)\Gamma(b)/\Gamma(a+b)$ is the Beta function
        and $\Gamma(x)$ is the Gamma function.
        Derive the mean, mode, and variance of $\theta$.
\end{problem}
\begin{solution}

\item Mean
\begin{equation}\begin{aligned}
E(x) & = \int \theta \PP(\theta; a,b) d\theta \\
& = \frac{1}{B(a,b)} \int \theta \theta^{a-1}(1-\theta)^{b-1} d\theta \\
& = \frac{1}{B(a,b)} \int \theta^a(1-\theta)^{b-1} d\theta \\
& = \frac{1}{B(a,b)} \int (1-\theta)^{b-1} \theta^a d\theta \\
\text{(by parts)} & = \frac{1}{B(a,b)} [ (1-\theta)^{b-1} \frac{\theta^{a+1}}{a + 1} ) + \frac{b-1}{a+1} \int \theta^{a+1} (1-\theta)^{b-2} d\theta ] \\
\end{aligned}\end{equation}    

Or maybe mathematica.  Closed form:
\begin{equation}\begin{aligned}
\int t B_t(t,a,b) dt & = \frac{1}{2} (t^2 B_t(t, a, b) - B_t(t, 2 + a, b) )
\end{aligned}\end{equation}

since $B(0,a,b) = 0$ and $B(1,a,b)$ is a constant in terms of the Gamma function
\begin{equation}\begin{aligned}
E(x) & = \frac{1}{2} ( B(a, b) - B(2 + a, b) ) \\
& = \frac{1}{2} ( \frac{\Gamma(a)\Gamma(b)}{\Gamma(a+b)} - \frac{\Gamma(2 + a)\Gamma(b)}{\Gamma(a+b + 1)} ) \\
& = \frac{\Gamma(b)}{2} ( \frac{\Gamma(a)}{\Gamma(a+b)} - \frac{\Gamma(2+a)}{\Gamma(2+a+b)} )
\end{aligned}\end{equation}

\newpage
\item Mode

  The Mode occurs where $B(t,a,b)$ has a maximum, so
\begin{equation}\begin{aligned}
    \frac{\partial}{\partial t} B(t,a,b) & = 0 \\
    \frac{\partial}{\partial t} t^{(a-1)}(1-t)^{b-1} & = 0 \\
    (a-1)t^{a-2}(1-t)^{b-1} - (b-2)t^{a-1}(1-t)^{b-2} & = 0 \\
    (a-1)(1-t) - (b-2)t & = 0 \\
    a + t - at - 1 - bt + 2t & = 0 \\
    (a + b - 2)t - (a - 1) & = 0 \\
    & \text{when  } t^* = \frac{a - 1}{a + b - 2}
\end{aligned}\end{equation}

can be evaluated by subsituting $t^*$ into the definition of $B_t$

\item Variance

Variance is $E[x^{2}] - (E[x])^{2}$, again with Mathematic

\begin{equation}\begin{aligned}
E[x^2] & = \int \theta^2 \PP(\theta; a,b) d\theta \\
& = \frac{1}{B(a,b)} \int \theta^2 \theta^{a-1}(1-\theta)^{b-1} d\theta \\
& = \frac{1}{3}( t^3 B_t(t,a,b) - B_t(t,3+a,b) )
\end{aligned}\end{equation}

so 
\begin{equation}\begin{aligned}
E[x^2] = \frac{1}{3} ( B(a,b) - B(3+a, b) )
\end{aligned}\end{equation}

From Mean, we have
\begin{equation}\begin{aligned}
E[x]^2  & = (\frac{1}{2} ( B(a,b) - B(2 + a, b) ))^2 \\
 & = \frac{1}{4} ( B(a,b)^2 + B(2 + a, b)^2 - 2B(a,b)B(2+a,b) ) \\
\end{aligned}\end{equation}

and Variance is
\begin{equation}\begin{aligned}
    Variance & = E[x^2] - E[x]^2 & = \\
& = \frac{1}{3} ( B(a,b) - B(3+a, b) ) - \frac{1}{4} ( B(a,b)^2 + B(2 + a, b)^2 - 2B(a,b)B(2+a,b) )
\end{aligned}\end{equation}


\vfill
\end{solution}
\newpage

\begin{problem}[2]
(\textbf{Murphy 9}) Show that the multinoulli distribution
\[
    \text{Cat}(\xx|\mub) = \prod_{i=1}^K \mu_i^{x_i}
\]
is in the exponential family and show that the generalized linear model
corresponding to this distribution is the same as multinoulli logistic
regression (softmax regression).
\end{problem}
\begin{solution}

To be in the exponential family, we must show:
\begin{equation}\begin{aligned}
f_X(x|\theta) = h(x) \exp[ \eta(\theta) \cdot T(x) - A(\theta) ]
\end{aligned}\end{equation}

\begin{equation}\begin{aligned}
    h(x) & = 1 \\
    \eta(\theta) & = [ ... \log \mu_i ... ] \\
    T(x) & = [ ... x_i ... ] \\
    A(x) & = 0
\end{aligned}\end{equation}



NLL of the above is linear in $x_i \log \mu_i$ 

\vfill
\end{solution}
\newpage

\end{document}
