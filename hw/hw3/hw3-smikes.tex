\documentclass[12pt,letterpaper,fleqn]{hmcpset}
\usepackage[margin=1in]{geometry}
\usepackage{graphicx}
\usepackage{amsmath,amssymb}
\usepackage{enumerate}
\usepackage{hyperref}
\usepackage{parskip}

% Theorems
\usepackage{amsthm}
\renewcommand\qedsymbol{$\blacksquare$}
\makeatletter
\@ifclassloaded{article}{
    \newtheorem{definition}{Definition}[section]
    \newtheorem{example}{Example}[section]
    \newtheorem{theorem}{Theorem}[section]
    \newtheorem{corollary}{Corollary}[theorem]
    \newtheorem{lemma}{Lemma}[theorem]
}{
}
\makeatother

% Random Stuff
\setlength\unitlength{1mm}

\newcommand{\insertfig}[3]{
\begin{figure}[htbp]\begin{center}\begin{picture}(120,90)
\put(0,-5){\includegraphics[width=12cm,height=9cm,clip=]{#1.eps}}\end{picture}\end{center}
\caption{#2}\label{#3}\end{figure}}

\newcommand{\insertxfig}[4]{
\begin{figure}[htbp]
\begin{center}
\leavevmode \centerline{\resizebox{#4\textwidth}{!}{\input
#1.pstex_t}}
\caption{#2} \label{#3}
\end{center}
\end{figure}}

\long\def\comment#1{}

\newcommand\norm[1]{\left\lVert#1\right\rVert}
\DeclareMathOperator*{\argmin}{arg\,min}
\DeclareMathOperator*{\argmax}{arg\,max}

% bb font symbols
\newfont{\bbb}{msbm10 scaled 700}
\newcommand{\CCC}{\mbox{\bbb C}}

\newfont{\bbf}{msbm10 scaled 1100}
\newcommand{\CC}{\mbox{\bbf C}}
\newcommand{\PP}{\mbox{\bbf P}}
\newcommand{\RR}{\mbox{\bbf R}}
\newcommand{\QQ}{\mbox{\bbf Q}}
\newcommand{\ZZ}{\mbox{\bbf Z}}
\renewcommand{\SS}{\mbox{\bbf S}}
\newcommand{\FF}{\mbox{\bbf F}}
\newcommand{\GG}{\mbox{\bbf G}}
\newcommand{\EE}{\mbox{\bbf E}}
\newcommand{\NN}{\mbox{\bbf N}}
\newcommand{\KK}{\mbox{\bbf K}}
\newcommand{\KL}{\mbox{\bbf KL}}

% Vectors
\renewcommand{\aa}{{\bf a}}
\newcommand{\bb}{{\bf b}}
\newcommand{\cc}{{\bf c}}
\newcommand{\dd}{{\bf d}}
\newcommand{\ee}{{\bf e}}
\newcommand{\ff}{{\bf f}}
\renewcommand{\gg}{{\bf g}}
\newcommand{\hh}{{\bf h}}
\newcommand{\ii}{{\bf i}}
\newcommand{\jj}{{\bf j}}
\newcommand{\kk}{{\bf k}}
\renewcommand{\ll}{{\bf l}}
\newcommand{\mm}{{\bf m}}
\newcommand{\nn}{{\bf n}}
\newcommand{\oo}{{\bf o}}
\newcommand{\pp}{{\bf p}}
\newcommand{\qq}{{\bf q}}
\newcommand{\rr}{{\bf r}}
\renewcommand{\ss}{{\bf s}}
\renewcommand{\tt}{{\bf t}}
\newcommand{\uu}{{\bf u}}
\newcommand{\ww}{{\bf w}}
\newcommand{\vv}{{\bf v}}
\newcommand{\xx}{{\bf x}}
\newcommand{\yy}{{\bf y}}
\newcommand{\zz}{{\bf z}}
\newcommand{\0}{{\bf 0}}
\newcommand{\1}{{\bf 1}}

% Matrices
\newcommand{\Ab}{{\bf A}}
\newcommand{\Bb}{{\bf B}}
\newcommand{\Cb}{{\bf C}}
\newcommand{\Db}{{\bf D}}
\newcommand{\Eb}{{\bf E}}
\newcommand{\Fb}{{\bf F}}
\newcommand{\Gb}{{\bf G}}
\newcommand{\Hb}{{\bf H}}
\newcommand{\Ib}{{\bf I}}
\newcommand{\Jb}{{\bf J}}
\newcommand{\Kb}{{\bf K}}
\newcommand{\Lb}{{\bf L}}
\newcommand{\Mb}{{\bf M}}
\newcommand{\Nb}{{\bf N}}
\newcommand{\Ob}{{\bf O}}
\newcommand{\Pb}{{\bf P}}
\newcommand{\Qb}{{\bf Q}}
\newcommand{\Rb}{{\bf R}}
\newcommand{\Sb}{{\bf S}}
\newcommand{\Tb}{{\bf T}}
\newcommand{\Ub}{{\bf U}}
\newcommand{\Wb}{{\bf W}}
\newcommand{\Vb}{{\bf V}}
\newcommand{\Xb}{{\bf X}}
\newcommand{\Yb}{{\bf Y}}
\newcommand{\Zb}{{\bf Z}}

% Calligraphic
\newcommand{\Ac}{{\cal A}}
\newcommand{\Bc}{{\cal B}}
\newcommand{\Cc}{{\cal C}}
\newcommand{\Dc}{{\cal D}}
\newcommand{\Ec}{{\cal E}}
\newcommand{\Fc}{{\cal F}}
\newcommand{\Gc}{{\cal G}}
\newcommand{\Hc}{{\cal H}}
\newcommand{\Ic}{{\cal I}}
\newcommand{\Jc}{{\cal J}}
\newcommand{\Kc}{{\cal K}}
\newcommand{\Lc}{{\cal L}}
\newcommand{\Mc}{{\cal M}}
\newcommand{\Nc}{{\cal N}}
\newcommand{\Oc}{{\cal O}}
\newcommand{\Pc}{{\cal P}}
\newcommand{\Qc}{{\cal Q}}
\newcommand{\Rc}{{\cal R}}
\newcommand{\Sc}{{\cal S}}
\newcommand{\Tc}{{\cal T}}
\newcommand{\Uc}{{\cal U}}
\newcommand{\Wc}{{\cal W}}
\newcommand{\Vc}{{\cal V}}
\newcommand{\Xc}{{\cal X}}
\newcommand{\Yc}{{\cal Y}}
\newcommand{\Zc}{{\cal Z}}

% Bold greek letters
\newcommand{\alphab}{\hbox{\boldmath$\alpha$}}
\newcommand{\betab}{\hbox{\boldmath$\beta$}}
\newcommand{\gammab}{\hbox{\boldmath$\gamma$}}
\newcommand{\deltab}{\hbox{\boldmath$\delta$}}
\newcommand{\etab}{\hbox{\boldmath$\eta$}}
\newcommand{\lambdab}{\hbox{\boldmath$\lambda$}}
\newcommand{\epsilonb}{\hbox{\boldmath$\epsilon$}}
\newcommand{\nub}{\hbox{\boldmath$\nu$}}
\newcommand{\mub}{\hbox{\boldmath$\mu$}}
\newcommand{\zetab}{\hbox{\boldmath$\zeta$}}
\newcommand{\phib}{\hbox{\boldmath$\phi$}}
\newcommand{\psib}{\hbox{\boldmath$\psi$}}
\newcommand{\thetab}{\hbox{\boldmath$\theta$}}
\newcommand{\taub}{\hbox{\boldmath$\tau$}}
\newcommand{\omegab}{\hbox{\boldmath$\omega$}}
\newcommand{\xib}{\hbox{\boldmath$\xi$}}
\newcommand{\sigmab}{\hbox{\boldmath$\sigma$}}
\newcommand{\pib}{\hbox{\boldmath$\pi$}}
\newcommand{\rhob}{\hbox{\boldmath$\rho$}}

\newcommand{\Gammab}{\hbox{\boldmath$\Gamma$}}
\newcommand{\Lambdab}{\hbox{\boldmath$\Lambda$}}
\newcommand{\Deltab}{\hbox{\boldmath$\Delta$}}
\newcommand{\Sigmab}{\hbox{\boldmath$\Sigma$}}
\newcommand{\Phib}{\hbox{\boldmath$\Phi$}}
\newcommand{\Pib}{\hbox{\boldmath$\Pi$}}
\newcommand{\Psib}{\hbox{\boldmath$\Psi$}}
\newcommand{\Thetab}{\hbox{\boldmath$\Theta$}}
\newcommand{\Omegab}{\hbox{\boldmath$\Omega$}}
\newcommand{\Xib}{\hbox{\boldmath$\Xi$}}

% mixed symbols
\newcommand{\sinc}{{\hbox{sinc}}}
\newcommand{\diag}{{\hbox{diag}}}
\renewcommand{\det}{{\hbox{det}}}
\newcommand{\trace}{{\hbox{tr}}}
\newcommand{\tr}{\trace}
\newcommand{\sign}{{\hbox{sign}}}
\renewcommand{\arg}{{\hbox{arg}}}
\newcommand{\var}{{\hbox{var}}}
\newcommand{\cov}{{\hbox{cov}}}
\renewcommand{\Re}{{\rm Re}}
\renewcommand{\Im}{{\rm Im}}
\newcommand{\eqdef}{\stackrel{\Delta}{=}}
\newcommand{\defines}{{\,\,\stackrel{\scriptscriptstyle \bigtriangleup}{=}\,\,}}
\newcommand{\<}{\left\langle}
\renewcommand{\>}{\right\rangle}
\newcommand{\Psf}{{\sf P}}
\newcommand{\T}{\top}
\newcommand{\m}[1]{\begin{bmatrix} #1 \end{bmatrix}}


% info for header block in upper right hand corner
\name{Sam Mikes}
\class{Math189R SU20.1}
\assignment{Homework 3}
\duedate{Thursday, June 18, 2019}

\begin{document}

\begin{problem}[1]
(\textbf{Murphy 2.16}) Suppose $\theta \sim \text{Beta}(a,b)$ such
        that
        \[
            \PP(\theta; a,b) = \frac{1}{B(a,b)} \theta^{a-1}(1-\theta)^{b-1} = \frac{\Gamma(a+b)}{\Gamma(a)\Gamma(b)} \theta^{a-1}(1-\theta)^{b-1}
        \]
        where $B(a,b) = \Gamma(a)\Gamma(b)/\Gamma(a+b)$ is the Beta function
        and $\Gamma(x)$ is the Gamma function.
        Derive the mean, mode, and variance of $\theta$.
\end{problem}
\begin{solution}

\item Mean
\begin{equation}\begin{aligned}
E(x) & = \int \theta \PP(\theta; a,b) d\theta \\
& = \frac{1}{B(a,b)} \int \theta \theta^{a-1}(1-\theta)^{b-1} d\theta \\
& = \frac{1}{B(a,b)} \int \theta^a(1-\theta)^{b-1} d\theta \\
& = \frac{1}{B(a,b)} \int (1-\theta)^{b-1} \theta^a d\theta \\
\text{(by parts)} & = \frac{1}{B(a,b)} [ (1-\theta)^{b-1} \frac{\theta^{a+1}}{a + 1} ) + \frac{b-1}{a+1} \int \theta^{a+1} (1-\theta)^{b-2} d\theta ] \\
\end{aligned}\end{equation}    

Or maybe mathematica.  Closed form:
\begin{equation}\begin{aligned}
\int t B_t(t,a,b) dt & = \frac{1}{2} (t^2 B_t(t, a, b) - B_t(t, 2 + a, b) )
\end{aligned}\end{equation}

since $B(0,a,b) = 0$ and $B(1,a,b)$ is a constant in terms of the Gamma function
\begin{equation}\begin{aligned}
E(x) & = \frac{1}{2} ( B(a, b) - B(2 + a, b) ) \\
& = \frac{1}{2} ( \frac{\Gamma(a)\Gamma(b)}{\Gamma(a+b)} - \frac{\Gamma(2 + a)\Gamma(b)}{\Gamma(a+b + 1)} ) \\
& = \frac{\Gamma(b)}{2} ( \frac{\Gamma(a)}{\Gamma(a+b)} - \frac{\Gamma(2+a)}{\Gamma(2+a+b)} )
\end{aligned}\end{equation}

\newpage
\item Mode

  The Mode occurs where $B(t,a,b)$ has a maximum, so
\begin{equation}\begin{aligned}
    \frac{\partial}{\partial t} B(t,a,b) & = 0 \\
    \frac{\partial}{\partial t} t^{(a-1)}(1-t)^{b-1} & = 0 \\
    (a-1)t^{a-2}(1-t)^{b-1} - (b-2)t^{a-1}(1-t)^{b-2} & = 0 \\
    (a-1)(1-t) - (b-2)t & = 0 \\
    a + t - at - 1 - bt + 2t & = 0 \\
    (a + b - 2)t - (a - 1) & = 0 \\
    & \text{when  } t^* = \frac{a - 1}{a + b - 2}
\end{aligned}\end{equation}

can be evaluated by subsituting $t^*$ into the definition of $B_t$

\item Variance

Variance is $E[x^{2}] - (E[x])^{2}$, again with Mathematic

\begin{equation}\begin{aligned}
E[x^2] & = \int \theta^2 \PP(\theta; a,b) d\theta \\
& = \frac{1}{B(a,b)} \int \theta^2 \theta^{a-1}(1-\theta)^{b-1} d\theta \\
& = \frac{1}{3}( t^3 B_t(t,a,b) - B_t(t,3+a,b) )
\end{aligned}\end{equation}

so 
\begin{equation}\begin{aligned}
E[x^2] = \frac{1}{3} ( B(a,b) - B(3+a, b) )
\end{aligned}\end{equation}

From Mean, we have
\begin{equation}\begin{aligned}
E[x]^2  & = (\frac{1}{2} ( B(a,b) - B(2 + a, b) ))^2 \\
 & = \frac{1}{4} ( B(a,b)^2 + B(2 + a, b)^2 - 2B(a,b)B(2+a,b) ) \\
\end{aligned}\end{equation}

and Variance is
\begin{equation}\begin{aligned}
    Variance & = E[x^2] - E[x]^2 & = \\
& = \frac{1}{3} ( B(a,b) - B(3+a, b) ) - \frac{1}{4} ( B(a,b)^2 + B(2 + a, b)^2 - 2B(a,b)B(2+a,b) )
\end{aligned}\end{equation}


\vfill
\end{solution}
\newpage

\begin{problem}[2]
(\textbf{Murphy 9}) Show that the multinoulli distribution
\[
    \text{Cat}(\xx|\mub) = \prod_{i=1}^K \mu_i^{x_i}
\]
is in the exponential family and show that the generalized linear model
corresponding to this distribution is the same as multinoulli logistic
regression (softmax regression).
\end{problem}
\begin{solution}

To be in the exponential family, we must show:
\begin{equation}\begin{aligned}
f_X(x|\theta) = h(x) \exp[ \eta(\theta) \cdot T(x) - A(\theta) ]
\end{aligned}\end{equation}

\begin{equation}\begin{aligned}
    h(x) & = 1 \\
    \eta(\theta) & = [ ... \log \mu_i ... ] \\
    T(x) & = [ ... x_i ... ] \\
    A(x) & = 0
\end{aligned}\end{equation}



NLL of the above is linear in $x_i \log \mu_i$ 

\vfill
\end{solution}
\newpage

\end{document}
